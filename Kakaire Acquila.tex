\documentclass[10pt,a4paper]{report}
\usepackage[latin1]{inputenc}
\usepackage{amsfonts}
\usepackage{graphicx}
\usepackage[width=5.00cm, left=2.00cm, right=2.00cm, top=2.00cm]{geometry}
\begin{document}
	\begin{titlepage}
		\centering
		\begin{figure}
			\centering
			\includegraphics[width=0.9\linewidth]{../../Desktop/Capture}
		\end{figure}
	\title{\textbf{SPENDING IN CAMPUS}
	\\How to be financially independent}
		\vfill
		\author{Acquila A. Kakaire
		\\B.Sc Computer Science
		\vspace{1.5cm}
		\\\textbf{14/U/6840/EVE}}
		
	\maketitle
	\end{titlepage}
	\section*{Introduction}
	This report shall cover how one can spend less while at campus and yet remain contented. This is due to the fact that many students have a deficit budget that has to cover all their welfare. This ranges from meals, transport, stationery, et cetera.
	Therefore, the sole reason is to make sure that we are all comfortable in the campus society, regardless of our social classes.
	\section*{Findings}
	We shall quickly highlight a few and I derive all this from my friends who live in campus, and a few from personal experiences.
	\paragraph{1. Saving}
	We can achieve financial stability through saving more. For example, you could save just shs. 1,000 from your initial expenditure daily. By the end of the month, you will realise you have shs. 30,000 (by the end of the semester, you have shs. 120,000, which you can use to buy an asset).
	\paragraph{2. Recheck your budget.}
	Frankly, our budgets are way too high, with things so irrelevant. Let the budget be comprised of basic things that you really can?t do without. 
	\paragraph{3. Lower your standards.}
	This is a tricky one, we forget what brought us to campus and we start to compete with other people?s lifestyles. This ranges from behaviour, to even getting into a relationship, just to please a friend(s). And we may not afford this, to be frankly.
	\paragraph{4. Make friends.}
	These can be a reliable resource when things backfire, like when you need coursework done but you don?t have a pc. They can also help with the stationery, but make sure you don?t overdo it.
	\paragraph{5. Visiting friends.}
	Let?s not take friends for granted, mostly those who are sleeping within campus. Friends come in handy when you are down and you just don?t know where the next meal will come from. As they say; a friend in need is a friend in need, but the trick here, make a rota. You will not be noticed.
	\paragraph{6. Fluking meals at the restaurants.}
	You could visit your nearest restaurant and move out without paying. But this comes at a cost, you need to be able to act, and it?s not for cowards. Free things come at a cost.
	\paragraph{7. Get into a relationship.}
	Specifically for girls, you can get into a relationship with benefits. This way, you will never lack, when it comes to those materialistic needs.
	\\
	I really think that we all need happiness, but getting out of our way can really be costly. With the above stated points, we should be able to achieve whatever brought us to campus, but for a good cause, I hope. And it all starts with us embracing our image in society and find a way to live that way. 
	\section*{Conclusion}
	This report has all content gathered from peers and experience, and therefore it \textit{isn?t} a guarantees will work for you, but simply take on those that you think you can afford to do. 
\end{document}