\documentclass[10pt,a4paper]{report}
\usepackage[latin1]{inputenc}
\usepackage{amsfonts}
\usepackage{graphicx}
\usepackage[width=5.00cm, left=2.00cm, right=2.00cm, top=2.00cm]{geometry}
\begin{document}
	\begin{titlepage}
		\centering
		\begin{figure}
			\centering
			\includegraphics[width=0.9\linewidth]{../../Desktop/Capture}
		\end{figure}
	\title{\textbf{TREE CONSERVATION REPORT}}
		\vfill
		\author{Acquila A. Kakaire
		\\B.Sc Computer Science
		\vspace{1.5cm}
		\\\textbf{14/U/6840/EVE}}
		
	\maketitle
	\end{titlepage}

\section{Introduction}
With the rampant rate of defforestation worldwide, the need to conserve trees that are a habitat to wildlife is key. A report has therefore been formulated to collect the species of birds that are in danger of extinction due to destruction of their natural habitats.

\section{Data Form}
This form will be able usd to collect data on the trees, their condition and in the end, they can be used get a clear picture of how  best we can be able to retain bird species.

\subsection{English name of tree}
This field will basically collect the name of the tree as is it commonly known in english.

\subsection {Botanical name of the tree}
This entry accepts the botanical name of the tree, that is, to specify the family and the species it belongs to.

\subsection {Tree Condition}
This will identify the condition of the tree, if it is in good condition or in a bad condition that may lead to its destruction.

\subsection


Added for your viewing convenience is a continuous preview mode for the PDF. This mode is enabled by default, but can also be disabled through the \emph{(View $\rightarrow$ Page layout in preview)} menu. Complementary to this feature is SyncTeX integration, which allows you to synchronize the position in your editor with the PDF preview. 

\section{Feedback}
We hope you will enjoy using this release as much as we enjoyed creating it. If you have comments, suggestions or wish to report an issue you are experiencing - contact us at: \emph{http://gummi.midnightcoding.org}.

\section{}


\end{document}
