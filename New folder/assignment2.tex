\documentclass[10pt,a4paper]{article}
\usepackage[latin1]{inputenc}
\usepackage{amsfonts}
\usepackage{graphicx}
\usepackage[width=6.00cm, left=1.00cm, right=1.00cm, top=2.00cm]{geometry}
\begin{document}
\begin{center}
\includegraphics[scale=1]{../muk.JPG}   \vfill
  {\Huge \textbf{TREE CONSERVATION REPORT}}
  \vspace{5cm}
  \\{\LARGE Acquila A. Kakaire}
  \vspace{1.5cm}
  \\{\Large B.Sc Computer Science}
  \vspace{1.5cm}
  \\{\Large 14/U/6840/EVE}
  \date{\today}
  \vspace{7.5cm}
		
\end{center}
\newpage

\section{Introduction}
With the rampant rate of deforestation worldwide, the need to conserve trees that are a habitat to wildlife is key. A report has therefore been formulated to collect the species of birds that are in danger of extinction due to destruction of their natural habitats.

\section{Data Form}
This form will be able used to collect data on the trees, their condition and in the end, they can be used get a clear picture of how  best we can be able to retain bird species.

\subsection{English name of tree}
This field will basically collect the name of the tree as is it commonly known in English.

\subsection {Botanical name of the tree}
This entry accepts the botanical name of the tree, that is, to specify the family and the species it belongs to.

\subsection {Tree Condition}
This will identify the condition of the tree, if it is in good condition or in a bad condition that may lead to its destruction.

\subsection {Input GPS location of study}
This will accept the location where the observation is being collected from.

\subsection {Does tree have a nest?}
This mandatory field will be an indicator of bird-life inhabiting the tree, accepting Yes/No values.

\subsubsection {Image of nest}
This field will be prompted only if one selects the Yes field in the previous field. It accepts the picture of the  nest.


\section{Sample of data}
Added for your viewing convenience is an example of the data collected.
\begin{flushleft}
\includegraphics[scale=0.1]{../Screenshot_2018-02-24-17-53-54.png}
\end{flushleft}
  
\end{document}
