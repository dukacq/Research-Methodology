\documentclass[10pt,a4paper]{article}
\usepackage[latin1]{inputenc}
\usepackage{amsfonts}
\usepackage{graphicx}
\usepackage[width=6.00cm, left=1.00cm, right=1.00cm, top=2.00cm]{geometry}
\begin{document}
\begin{center}
\includegraphics[scale=1]{../muk.JPG}   \vfill
  {\Huge \textbf{TREE CONSERVATION REPORT}}
  \vspace{5cm}
  \\{\LARGE Acquila A. Kakaire}
  \vspace{1.5cm}
  \\{\Large B.Sc Computer Science}
  \vspace{1.5cm}
  \\{\Large 14/U/6840/EVE}
  \date{\today}
  \vspace{7.5cm}
		
\end{center}
\newpage

\section{Introduction}
With the rampant rate of deforestation worldwide, the need to conserve tree species is fundamental in the world we live in order to have a healthy environment, since different tree kinds have different uses.

A report has been formulated to have a count of tree species in a given location, and their condition. The results of this report will help sensitise the locals of the specified areas on which trees they should strive to conserve.

The data will be collected manually using the form that I have designed and I will be explaining what data tools will be used to collect information, and how.

\section{Data Form}
This form will be able used to collect data on the trees, their condition and in the end, they can be used get a clear picture of how  best we can be able to retain bird species.

\subsection{English name of tree}
This field will basically collect the name of the tree as is it commonly known in English.

\subsection {Botanical name of the tree}
This entry accepts the botanical name of the tree, that is, to specify the family and the species it belongs to.

\subsection{Approximate Tree Height in metres}
This field will take the approximate tree height as viewed by the observer.

\subsection {Input GPS location of study}
This will accept the location where the observation is being collected from.

\subsection {Tree Condition}
This will identify the condition of the tree, if it is in good condition or in a bad condition that may lead to its destruction.

\subsection {Image of Tree}
This field will be prompted to confirm the tree condition and the height.


\section{Sample of data}
Added for your viewing convenience is an example of the data collected.
\begin{flushleft}
\includegraphics[scale=0.9]{Capture.JPG} 
\end{flushleft}
  
\end{document}
